\documentclass{article}

\title{Every square can be tiled with T-tetrominos and no more than 5 monominos}
\author{Jack Grahl}

\begin{document}
\maketitle

\begin{abstract}
If $n$ is a multiple of 4, then a square of side $n$ can be tiled with T-tetrominos, using a well-known construction.
If $n$ is even but not a multiple of four, then there exists an equally well-known construction for tiling a square of side $n$ with T-tetrominos and exactly 4 monominos.
On the other hand, it was shown by Walkup in \cite{walkup} that it is not possible to tile the square using only T-tetrominos.
Now consider the remaining cases, where $n$ is odd.
It was shown by Zhan in \cite{zhan} that it is not possible to tile such a square using only one monomino.
Hochberg showed in \cite{hochberg} that no more than 9 monominos are ever needed.
We give a construction for all odd $n$ which uses exactly 5 monominos, thereby resolving this question.
Hochberg also conjectured that no more than 5 monominos are needed for any rectangle, with rectangles of odd height and odd width being the only possibility for a counter-example.
Our construction can also be used for certain classes of such rectangles, and reduces the spaces of possible counter-examples.
\end{abstract}
\bibliography{polyomino}{}
\bibliographystyle{plain}
\end{document}
