\documentclass{article}

\usepackage{amsmath, amsfonts, amssymb, amsthm}
\usepackage{tikz}
\usetikzlibrary{patterns}

\theoremstyle{plain}
\newtheorem{theorem}{Theorem}[section]
\newtheorem{proposition}[theorem]{Proposition}
\newtheorem{lemma}[theorem]{Lemma}
\newtheorem{definition}[theorem]{Definition}

\title{Every rectangle can be tiled with T-tetrominos and no more than 5 monominos}
\author{Jack Grahl}

\begin{document}
\maketitle

\begin{abstract}
If $n$ is a multiple of 4, then a square of side $n$ can be tiled with T-tetrominos, using a well-known construction.
If $n$ is even but not a multiple of four, then there exists an equally well-known construction for tiling a square of side $n$ with T-tetrominos and exactly 4 monominos.
On the other hand, it was shown by Walkup in \cite{walkup} that it is not possible to tile the square using only T-tetrominos.
Now consider the remaining cases, where $n$ is odd.
It was shown by Zhan in \cite{zhan} that it is not possible to tile such a square using only one monomino.
Hochberg showed in \cite{hochberg} that no more than 9 monominos are ever needed.
We give a construction for all odd $n$ which uses exactly 5 monominos, thereby resolving this question.
Hochberg also conjectured that no more than 5 monominos are needed for any rectangle.
Our construction can also be used for some classes of rectangles with odd width and height, other than squares, and reduces the space of possible counter-examples.
\end{abstract}

\begin{theorem}
Every square can be tiled with T-tetrominos and at most 5 monominos.
\end{theorem}
This theorem follows immediately from propositions \ref{four}, \ref{even} and \ref{odd}.

\begin{proposition}\label{four}
Every square of side $n = 4m$ can be tiled with T-tetrominos.
\end{proposition}
\begin{proposition}\label{even}
Every square of side $n = 4m + 2$ can be tiled with T-tetrominos and 4 monominos, and 4 monominos are always needed.
\end{proposition}
For $n = 2$ this is the same as pointing out that a single T-tetromino will not fit in the $2x2$ square.

\begin{proposition}\label{odd}
Every square of side $n = 2m + 1$ can be tiled with T-tetrominos and 5 monominos, and 5 monominos are always needed (except for $n = 1$).
\end{proposition}
Zhan's (\cite{zhan}) Theorem 2 states that it is not possible to tile any rectangle with T-tetrominos and only one monomino. It must therefore be the case that at least 5 are needed. We show that exactly 5 are sufficient.

\begin{definition}
Call $A_n$ the set of lattice squares given by the square of side $n$, with the lattice squares at $(0, 0), (0,1), (1, 0)$ and $(0, n-1)$ removed.
This shape has area $n^2 - 4 = 4(m^2 + m - 1) + 1$.
\end{definition}

\begin{lemma}
For all $m \in \mathbb{N}$, $A_{2m+1}$ can be tiled with $m^2 + m - 1$ T-tetrominos and one monomino.
\end{lemma}

\begin{figure}
\input{cropped_image}
\caption{$A_5$, the $5 \times 5$ square with four lattice squares removed, $A_7$ and $A_9$.}
\label{cropped}
\end{figure}

\begin{figure}
\input{five_image}
\caption{Tiling of $A_5$ with a single monomino.}
\label{five}
\end{figure}

{\sc Proof.}
The proof is by induction on $n$. In figure \ref{five} we show how $A_5$ can be tiled by 5 tetrominos and a single monomino. (It is trivial to tile $A_3$ with a single tetromino and a single monomino, but it is slightly clearer to start the induction with $n=5$.) If $A_n$ can be tiled with one monomino, then so can $A_{n+1}$. There are two constructions for the cases $n=4k+1$ and $n=4k+3$.

\begin{figure}
\input{ones_image}
\caption{A tiling of $A_{4k+1}$ can be extended to a tiling of a reflected copy of $A_{4k+3}$.}
\label{ones}
\end{figure}

\begin{figure}
\input{threes_image}
\caption{A tiling of $A_{4k+3}$ can be extended to a tiling of a reflected copy of $A_{4(k+1)+1}$.}
\label{threes}
\end{figure}

\section{Tiling every rectangle}
\theorem{Every rectangle of side length at least 12 can be tiled with T-tetrominos and either 0, 2, 3, 4, 5 or 6 monominos.}
Clearly the number of monominos depends only on the number of grid squares modulo 4, in the case where this is odd.

In the terminology used by Hochberg, the \emph{gap cap} of the T-tetromino is no more than 6 (and it is known that it can be no less than 5).

This theorem is an immediate consequence of the following sequences of lemmata, most of which are either part of the folklore or given previously.

\begin{lemma}
Every rectangle of dimensions $4l \times 4m$ can be tiled without monominos.
\end{lemma}
This was stated by Walkup in \cite{walkup}, who simply pointed out that it is an obvious consequence of the fact that the square of side 4 can be tiled, which is itself easy to check.

\begin{lemma}
Every rectangle of dimensions $4l + 2 \times 4m + 2$ can be tiled with 4 monominos, and 4 are always needed.
\end{lemma}
The $2x2$ square can trivially not contain any T-tetromino.

\begin{figure}
\input{strips_image}
\caption{Tilings of strips of width 2 and length $4n + 1$, $4n+2$ and $4n+3$, with 2, 4 and 2 monominos respectively.}
\label{strips}
\end{figure}

Rectangles of this form, with exactly one side of length 2, can be tiled using the second construction in figure \ref{strips}. It is clear that the only other tilings are variations of this where some tiles are shifted along by 1 square, or reflections of these.

If both side lengths are at least $6$, Walkup's Theorem 1 (\cite{walkup}) shows that it is not possible to tile the square without monominos.
Clearly at least 4 are needed.
As Hochberg points out in Theorem 8 of \cite{hochberg}, a tiling with only tetrominos of a rectangle whose sides are both multiples of 4, can be extended to a tiling of a rectangle which is bigger by 2 in each direction, using only 4 monominos.
The L-shaped strip which is added to the former looks (for example) like Figure \ref{lshaped}.

\begin{figure}
\input{lshaped_image}
\caption{The L-shaped strip which extends any square of side $4n$ to a square of side $4n+2$, using exactly 4 monominos. The dotted tetrominos can be repeated as many times as necessary to give a strip with internal length $4n$ and external length $4n + 2$.}
\label{lshaped}
\end{figure}

\begin{lemma}
Every rectangle of dimensions $4l \times 4m + 2$ can be tiled with 4 monominos, and 4 are always needed.
\end{lemma}
Again, Walkup has shown that tilings without any monominos are not possible. Using the tiling of a strip of width 2 and length $4l$ given in figure \ref{strips}, we can extend a tiling of a $4l \times 4m$ rectangle which uses only T-tetrominos to a tiling of this rectangle which uses exactly 4 monominos.

\begin{lemma}
Fuck knows about rectangles $odd \times even$.
\end{lemma}



\bibliography{polyomino}{}
\bibliographystyle{plain}
\end{document}
